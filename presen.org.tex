\begin{outline-text-1}
\begin{xlarge}
\begin{center}
To Start Press "n"
\end{center}
\end{xlarge}

\begin{center}
\textbf{WARN} Turn off your custom keyboard shortcuts
\end{center}

\begin{note}
\begin{alignright}
Made by guicho2.71828 (Masataro Asai)
\end{alignright}
\end{note}
\end{outline-text-1}

\section{What Is It?}
\label{sec-1}

\begin{xlarge}
An Alternative to
\begin{center}
Org-based Presentation
\end{center}
\begin{alignright}
Method
\end{alignright}
\end{xlarge}

\begin{center}
Proceed with "n"
\end{center}

\section{Usage (on browsers)}
\label{sec-2}

\begin{smaller}
\begin{description}
\item[{n}] go to the next section/expand a list
\item[{p}] back to the previous section
\item[{s, or "go"}] jump to an arbitrary section (using minibuffer)
\item[{d}] debugging mode (shows content border)
\item[{<Ctrl-g> or <Esc>}] cancel current key strokes
\item[{"help"}] shows the available commands in the minibuffer!
\end{description}
\end{smaller}

\section{Usage (on emacs)}
\label{sec-3}

\begin{enumerate}
\item clone \href{https://github.com/guicho271828/another-org-info}{our repo}

\item \texttt{make html} to publish

\item show it on your browser!
\end{enumerate}

\section{Virtue}
\label{sec-4}

\begin{itemize}
\item Simplicity
\begin{itemize}
\item \texttt{code.js} is a pure javascript, only has 200 lines (increasing as new features
are added)
\end{itemize}
\item Extensibility
\begin{itemize}
\item you can write your own extension and its easy
\item set a function to \texttt{keyManager.<your-key-strokes>}
\item the key strokes should contain character \(c\) within char code \(32 < c < 126\)
\begin{itemize}
\item \texttt{Space,!,",...A,B,C,...x,y,z,\{,|,\textasciitilde{}}
\end{itemize}
\end{itemize}
\item LICENSE
\begin{itemize}
\item GPLv3 ?? < not attatched a license yet
\end{itemize}
\end{itemize}

\section{Example: Expanding a List}
\label{sec-5}

\begin{itemize}
\item \textbf{the dots} tell "\emph{here are more contents}" \(\rightarrow\)
\item \textbf{an arrow} tells "\emph{here are expandable contents}" \(\rightarrow\)
\begin{itemize}
\item These remind you that there are
\item \textbf{still more sub-contents} you have to talk
\item and where it ends
\end{itemize}
\item list
\item list
\end{itemize}


\section{Org-block templates}
\label{sec-6}

We mark the html tag with org-mode blocks such as \texttt{\#+BEGIN\_EXAMPLE}.
Using \href{http://orgmode.org/manual/Easy-Templates.html}{block template} (by modifying \texttt{org-structure-template-alist}),
you can quickly write such blocks i.e.:

\begin{verbatim}
at the beginning of line, hit

<s<tab>

in normal org-mode destribution in emacs,
it would expand into:

#+BEGIN_SRC 

#+END_SRC
\end{verbatim}

\subsection{customizing templates}
\label{sec-6-1}

We provide custom \texttt{org-structure-template-alist}.
These blocks are recognized by the stylesheet we provide,
so they are highlighted correctly.

\begin{container-fluid}
\begin{row-fluid}
\begin{span6}
\begin{itemize}
\item <S -- smaller
\item <lar -- larger
\item <t -- twocolumn
\item <n -- footnote
\item <r -- float right
\item <ar -- align right
\item <lcr -- "left,center,right" format with x-large fonts
\item <E -- latex `equation' environment
\item still more\ldots{}
\end{itemize}
\end{span6}
\begin{span6}
copy-paste it
to your \texttt{.emacs}

\begin{center}
\url{config.el}
\end{center}
\end{span6}
\end{row-fluid}
\end{container-fluid}

\section{Images}
\label{sec-7}

\begin{itemize}
\item Put images into \texttt{img/} directory.
\item svg files are converted into png.
\begin{itemize}
\item We recommend it because
\begin{enumerate}
\item In html presentation, the inserted svg images capture the keyboard and mouse event
and the presentation sometimes does not work right.
\item In resume output and pdf presentation, the \texttt{svg} and \texttt{svg-inkscape} package is a
bit buggy and I personally failed make it work.
\end{enumerate}
\end{itemize}
\item svg-png conversion is done by \texttt{inkscape}
\begin{itemize}
\item but you can also use ImageMagick \texttt{convert} command.
\item See \texttt{img/Makefile} for more information.
\end{itemize}
\end{itemize}

\section{Dependencies}
\label{sec-8}
\label{dependency}

\begin{itemize}
\item jQuery (included)
\item Inkscape (image conversion)
\item org-mode (optional submodule)
\item MathJax (optional submodule)
\end{itemize}

\section{\LaTeX{} output}
\label{sec-9}

\begin{itemize}
\item just \texttt{make pdf} on your shell
\item need the latest \$\TeX\$live installation
\begin{itemize}
\item see \texttt{Makefile}
\item \texttt{pdflatex}
\item alternatively, \texttt{platex} and \texttt{dvipdfmx}
\end{itemize}
\item presen.pdf -- presentation in pdf with similar appearance. Useful when
browser-based presentation is not allowed
\item resume.pdf -- short and printed version.
\end{itemize}

\section{Utility}
\label{sec-10}

\begin{itemize}
\item ./make-periotically.sh [args]
\begin{itemize}
\item Watches the changes in the directory and \texttt{make}
\item Build statuses are notified in inotify pop-up
\item all arguments are passed to \texttt{make}
\item dependency : inotifywait, notify-send
\end{itemize}
\end{itemize}

\section{Makefile Target}
\label{sec-11}

\begin{itemize}
\item \texttt{make html} -- builds the html version
\item \texttt{make resume} -- builds the resume version
\item \texttt{make pdf} -- builds the pdf version
\item \texttt{make} -- build all
\end{itemize}

\section{Test}
\label{sec-12}

\begin{itemize}
\item \href{http://www.google.com}{Link}
\item This
\item Is
\item A Test
\end{itemize}

Mathjax formula:

\[
 E=mc^2
\]

\begin{equation}
 E=mc^2 + \frac{1}{2} mv^2
\end{equation}

\subsection{Twocolumn Test}
\label{sec-12-1}

\begin{container-fluid}
\begin{row-fluid}
\begin{span6}
\begin{itemize}
\item HOOA!
\item \textbf{HOOA!}
\item HOOA!
\end{itemize}
\end{span6}
\begin{span6}
This is a LISP ALIEN IN A CAGE!

\includegraphics[width=.9\linewidth]{img/alien.png}
\end{span6}
\end{row-fluid}
\end{container-fluid}


\subsection{many columns test}
\label{sec-12-2}


\begin{container-fluid}
\begin{row-fluid}
\begin{span3}
a a a a a a a a a a a a a a a a a a a a a a a a a a a a a a a a
\end{span3}
\begin{span3}
b b b b b b b b b b b b b b b b b b b b b b b b b b b b b b b b
\end{span3}
\begin{span3}
c c c c c c c c c c c c c c c c c c c c c c c c c c c c c c c c
\end{span3}
\begin{span3}
d d d d d d d d d d d d d d d d d d d d d d d d d d d d d d d d
\end{span3}
\end{row-fluid}
\end{container-fluid}


\section{A Slide with Too Little Contents}
\label{sec-13}

\begin{center}
\begin{smaller}
Hi, I'm small!
\end{smaller}
\end{center}

\begin{note}
See the headline is correctly adjusted
\end{note}

\section{Left-Center-Right template}
\label{sec-14}

\begin{xlarge}
x-large left
\begin{center}
centered
\end{center}
\begin{alignright}
right
\end{alignright}
\end{xlarge}

\begin{note}
This is a footnote
\end{note}

\section{TODOs}
\label{sec-15}


\begin{container-fluid}
\begin{row-fluid}
\begin{span6}
\begin{smaller}
\begin{itemize}
\item Features
\begin{itemize}
\item Table of contents
\item \texttt{<dl>} does not expand
\item \sout{Showing current keystrokes} \textbf{DONE}
\item auto-scroll/auto-zoom with big contents
\item Showing current/total page number
\item Changing Stylesheet
\item Up-Section command
\item Slide thumbnail
\item \sout{stopwatch/countdown timer} \textbf{DONE}
\item link to \#section
\end{itemize}
\end{itemize}
\end{smaller}
\end{span6}
\begin{span6}
\begin{smaller}
\begin{itemize}
\item Features inspired by other tools
\begin{itemize}
\item Content Search (in org-infojs)
\item Drawing mode (in \href{http://code.google.com/p/jessyink/}{jessyink})
\item 'Paused' mode (in \href{http://lab.hakim.se/reveal-js/}{reveal.js})
\item \sout{Export to PDF (also in reveal.js)}
\begin{itemize}
\item implemented as the resume output
\end{itemize}
\item Slide with an image covering entire background (slideshare)
\item present one paragraph/word/letter at a time
\begin{itemize}
\item those in \href{http://docutils.sourceforge.net/docs/user/slide-shows.s5.html}{s5}
\end{itemize}
\item "C-M-x" style notation in the command definition
\end{itemize}
\end{itemize}
\end{smaller}
\end{span6}
\end{row-fluid}
\end{container-fluid}
